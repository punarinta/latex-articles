\documentclass[12pt]{article}

\usepackage[parfill]{parskip}
\usepackage{mathtools}

\begin{document}

\title{Speed estimations for an unmounted movement through a terrain}
\author{Vladimir Osipov}
\date{July 2015}
\maketitle

\section{Intro}

The aim of this article is to design a convenient method for calculating infantryman movement speed by feet through an arbitrary terrain and to test it on practice. This article is planned as a part of a series on infantryman computer-aided system design. Parts of the data will be received theoretically based on earlier researches, parts will be measured directly in situ. Tests will be conveyed with the help of MilSim participants in conditions close to battlefield ones. As all the experiments are done in a boreal forests area, terrain types will be limited to those common to these territories.

\newpage

\section{Theoretical derivations}

\subsection{Terrain slowdown reasons}

Generally there are three main slowdown reasons during a passage through any terrain: placement of obstacles on the way (a), terrain flatness on microlevel (b) and terrain flatness on macrolevel (c). The border between microlevel and macrolevel is suggested to put to a value of step length, as when the obstacle is greater it is usually already a part of a terrain. Factors (a) and (b) can be joined into one characteristic, let us call it \textit{passability} $p$ and describe it below. Macrolevel flatness, in its turn, describes expenditures linked to altitude changes and is definitely a function of those changes. Let us call this a \textit{slope effect} $s$ and also discuss it below.

As passage time estimations are generally needed when talking about terrain, let us introduce a parameter called \textit{difficulty} $D$, which will be a default movement time multiplier in $t=t_{0}D$, where $t_{0}$ is an ideal terrain passage time and $t$ is a sample terrain passage time, both $t$ and $t_{0}$ are measured at the same energy expenditure rate. As movement through an ideal terrain \footnote{Where a path has zero grade and nothing obstructs a person from walking.} is fastest by definition let us assume that $D\geq1$ for a non-assisted motion.


\subsection{Passability impact}
Let us define passability as a ratio between the speed of movement through a plain terrain with given obstacles over the movement speed through a terrain with ideal conditions (a straight flat asphalt road could be a very good example):

\begin{equation} \label{eq:p-definition}
p = \upsilon_{t}/\upsilon_{0}
\end{equation}

Thus the impact of passability on difficulty may be described as $D\sim\frac{1}{p}$. Later we will be trying to experimentally find out passability values for different terrain types by measuring travel time ratio on a fixed distance: $p=t_{0}/t_{t}$.


\subsection{Slope effect}
Movements upwards and downwards differ. While hill descending for gentle slopes (i.e. slopes where friction is enough to prevent a foot from sliding down) has approximately the same energy expenditures and speeds as a usual flat terrain movement \cite{mets-in-exercise}, hill ascending is significantly more energy demanding, due to the necessity to increase body potential energy. Let us define slope effect as a ratio between a speed on an ideal terrain and a speed on a slope, so the greater the slope effect --- the lower the actual terrain speed.

\begin{equation} \label{eq:s-definition}
s = \upsilon_{0}/\upsilon_{t}
\end{equation}

Slope effect should thus contribute into difficulty proportionally: $D\sim{s}$, so that

\begin{equation} \label{eq:diff-se}
D=\frac{s}{p}
\end{equation}

The reason of slowdown is a necessity to consume more energy per unit of length compared to a situation with an ideal terrain. As it is comparatively difficult to measure slope movement parameters with a constant energy consumption, it is suggested to estimate them based on a data from similar researches.

Energy expenditures may be defined via metabolic equivalent (MET) of an activity, where 1 MET is equal to $4.184\frac{kJ}{kg\cdot{h}}$ \cite{wiki-met}. From \cite{acsm-guidelines} we can see that MET value depends only on velocity if the grade value is fixed and on contrary does not depend on a grade if velocity is constant, i.e. $\frac{MET-MET_{0}}{g} = const$.

Also from \cite{acsm-guidelines} we have

\begin{equation} \label{eq:met-slow}
MET=\frac{0.1\upsilon+0.018\upsilon g + 3.5}{3.5}
\end{equation}

for speeds less than 100 m/min\footnote{What equals to 6 km/h.} and

\begin{equation} \label{eq:met-fast}
MET=\frac{21.11 - 0.3593\upsilon + 0.03\upsilon^{2} + 0.018\upsilon g}{3.5}
\end{equation}

for greater ones. With $g_{0} = 0$ and $MET_{1} = MET_{0}$ formulas \ref{eq:s-definition} and \ref{eq:met-slow} give us

\begin{equation} \label{eq:s-slow}
s_{slow} = 0.18g + 1
\end{equation}

while \ref{eq:s-definition} and \ref{eq:met-fast} result in

\begin{equation} \label{eq:s-fast}
s_{fast} = \frac{0.03\upsilon_{2}-0.3593 + 0.018g}{0.03\upsilon_{1}-0.3593}
\end{equation}

where $g$ is a slope grade in percents. This approximation is applicable only in a usual human velocities range and for slopes where surface friction is enough to step without sliding down. Formula \ref{eq:s-fast} is not going to be used for quick slope effect estimations, only for computer aided ones. 

We do not measure $g$ value directly, but we may derive it knowing that by definition \cite{wiki-grade} it is:
\begin{equation} \label{eq:g-def}
g = \frac{H}{L} \cdot 100
\end{equation}


\subsection{Difficulty}
In this section we will combine data on passability and slope effect and will derive both a quick estimation formula and a full formula for machine calculations.

Quick estimation is intended to be done in mind by a person without computer aid available, using only a map with a known scale and height marks or isolines present. This is not going to have much application within the current project, but may be pretty useful for purposes like hiking or non-computer  path planning for the military. Quick estimation will work better for long routes as the assumptions it is based on are supposed to give better results with high length-to-height ratio.


\subsubsection{Quick estimation}
Earlier we have noted that $g = 100 H / L$. Let us use a linear approximation and take $H$ as an altitude difference between the highest and the lowest points on the route while $L$ will be a distance between two most distant points of the route. Although a route length in general case is always greater than its horizontal projection, we will assume for the sake of simplicity that they are equal. Thus $L$ will be the distance between aforementioned points. Then using \ref{eq:s-slow} and \ref{eq:g-def}:

\begin{equation*}
s = 18\cdot\frac{H}{L} + 1
\end{equation*}

Together with \ref{eq:diff-se} this gives us

\begin{equation*}
D = \frac{18 H + L}{pL}
\end{equation*}

Quick estimation value needs to be proven experimentally and adjusted based on the results.


\subsubsection{Exact value}
During a computer analysis of a terrain map or a GPS track one gets data on every single altitude change and therefore on slope $i$, as a grade level on this slope is by definition constant. Slope effect for every linear part of a path will simply be $s_{i}$. Difficulty on such a slope will be:

\begin{equation}
d_{i} = \frac{1}{p_{i}} \cdot (\frac{18 \Delta h_{i}}{l_{i}} + 1)
\end{equation}

The overall difficulty of the route is $D = \frac{1}{L} \int d_{i} dx$ or for numerical calculations:

\begin{equation}
D = \sum d_{i} / N_{steps}
\end{equation}


\newpage
\section{Measurements}

\subsection{Passability experiment}
As mentioned above we are going to measure terrain passability as $t_{0}/t$, where both measurements are done on a flat terrain. $t_{0}$ is a passage time of a distance $L$ on a straight asphalt road, $t$ is a time for an investigated terrain. The following terrain types will be chosen for experiment: soil road, grassy field, clear pine forest, pine forest with dense shrub layer, mixed forest. 


\subsection{Results discussion}



\newpage
\appendix
\bibliographystyle{plain}
\bibliography{biblio}

\end{document}
